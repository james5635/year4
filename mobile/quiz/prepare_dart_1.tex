\documentclass[11pt,a4paper]{article}
\usepackage[utf8]{inputenc}
\usepackage[margin=1in]{geometry}
\usepackage{listings}
\usepackage{xcolor}
\usepackage{enumitem}

% Dart code styling
\lstdefinestyle{dart}{
    language=Java,
    basicstyle=\ttfamily\small,
    keywordstyle=\color{blue}\bfseries,
    stringstyle=\color{red},
    commentstyle=\color{gray}\itshape,
    numbers=none,
    breaklines=true,
    frame=single,
    backgroundcolor=\color{gray!10}
}

\lstset{style=dart}

\title{\textbf{Dart Programming Quiz}}
\author{}
% \author{sou chanrojame}
\date{}

\begin{document}

\maketitle

\section*{Part 1: Multiple Choice Questions (QCM)}

\begin{enumerate}

	\item What is the correct way to declare a constant value in Dart?
	      \begin{enumerate}[label=\Alph*)]
		      \item \texttt{var pi = 3.14;}
		      \item \texttt{final pi = 3.14;}
		      \item \texttt{const pi = 3.14;}
		      \item \texttt{let pi = 3.14;}
	      \end{enumerate}

	\item Which of the following is NOT a valid Dart data type?
	      \begin{enumerate}[label=\Alph*)]
		      \item \texttt{int}
		      \item \texttt{double}
		      \item \texttt{decimal}
		      \item \texttt{String}
	      \end{enumerate}

	\item What will the following code print?
	      \begin{lstlisting}
void main() {
  var a = 5;
  var b = 2;
  print(a ~/ b);
}
\end{lstlisting}
	      \begin{enumerate}[label=\Alph*)]
		      \item \texttt{2.5}
		      \item \texttt{2}
		      \item \texttt{3}
		      \item \texttt{Error}
	      \end{enumerate}

	\item How do you define a named parameter in Dart?
	      \begin{enumerate}[label=\Alph*)]
		      \item \texttt{void greet(String name) \{\}}
		      \item \texttt{void greet(\{String name\}) \{\}}
		      \item \texttt{void greet([String name]) \{\}}
		      \item \texttt{void greet(*String name*) \{\}}
	      \end{enumerate}

	\item Which of the following is correct for creating a List in Dart?
	      \begin{enumerate}[label=\Alph*)]
		      \item \texttt{var list = [1, 2, 3];}
		      \item \texttt{var list = (1, 2, 3);}
		      \item \texttt{var list = \{1, 2, 3\};}
		      \item \texttt{var list = <1, 2, 3>;}
	      \end{enumerate}

	\item What is the output of this code?
	      \begin{lstlisting}
void main() {
  var name;
  print(name ?? "Guest");
}
\end{lstlisting}
	      \begin{enumerate}[label=\Alph*)]
		      \item \texttt{null}
		      \item \texttt{Guest}
		      \item \texttt{""} (empty string)
		      \item \texttt{Error}
	      \end{enumerate}

	\item What is the default value of an uninitialized \texttt{int} variable in Dart?
	      \begin{enumerate}[label=\Alph*)]
		      \item 0
		      \item \texttt{null}
		      \item undefined
		      \item 1
	      \end{enumerate}

	\item Which of the following is true about \texttt{final} in Dart?
	      \begin{enumerate}[label=\Alph*)]
		      \item Must be assigned at compile time
		      \item Can only be assigned once at runtime
		      \item Can be reassigned multiple times
		      \item Must be mutable
	      \end{enumerate}

	\item Which operator is used for \textbf{type casting} in Dart?
	      \begin{enumerate}[label=\Alph*)]
		      \item \texttt{as}
		      \item \texttt{is}
		      \item \texttt{is!}
		      \item \texttt{->}
	      \end{enumerate}

	\item What is the output?
	      \begin{lstlisting}
void main() {
  print(3 ~/ 2);
}
\end{lstlisting}
	      \begin{enumerate}[label=\Alph*)]
		      \item 1.5
		      \item 1
		      \item 2
		      \item Error
	      \end{enumerate}

	\item How do you declare an optional positional parameter?
	      \begin{enumerate}[label=\Alph*)]
		      \item \texttt{void f(int x)}
		      \item \texttt{void f([int x])}
		      \item \texttt{void f(\{int x\})}
		      \item \texttt{void f(*int x*)}
	      \end{enumerate}

	\item Which collection in Dart is \textbf{unordered and does not allow duplicates}?
	      \begin{enumerate}[label=\Alph*)]
		      \item \texttt{List}
		      \item \texttt{Map}
		      \item \texttt{Set}
		      \item \texttt{Queue}
	      \end{enumerate}

	\item How do you create a Map in Dart?
	      \begin{enumerate}[label=\Alph*)]
		      \item \texttt{var m = \{1, 2\}}
		      \item \texttt{var m = \{1: 'one', 2: 'two'\}}
		      \item \texttt{var m = [1: 'one', 2: 'two']}
		      \item \texttt{var m = Map(1,2)}
	      \end{enumerate}

	\item How do you check if a variable is null?
	      \begin{enumerate}[label=\Alph*)]
		      \item \texttt{x == null}
		      \item \texttt{x ? null}
		      \item \texttt{x ?? null}
		      \item \texttt{x.isNull()}
	      \end{enumerate}

	\item Which keyword is used to create an asynchronous function?
	      \begin{enumerate}[label=\Alph*)]
		      \item \texttt{async}
		      \item \texttt{await}
		      \item \texttt{future}
		      \item \texttt{yield}
	      \end{enumerate}

	\item What is the output?
	      \begin{lstlisting}
void main() {
  List<int> l = [1,2,3];
  print(l.contains(2));
}
\end{lstlisting}
	      \begin{enumerate}[label=\Alph*)]
		      \item 0
		      \item true
		      \item 2
		      \item false
	      \end{enumerate}

	\item How do you define a \textbf{constant constructor} in a class?
	      \begin{enumerate}[label=\Alph*)]
		      \item \texttt{const ClassName()}
		      \item \texttt{ClassName.const()}
		      \item \texttt{final ClassName()}
		      \item \texttt{static ClassName()}
	      \end{enumerate}

	\item Which of the following is a \textbf{null-aware operator} in Dart?
	      \begin{enumerate}[label=\Alph*)]
		      \item \texttt{?.}
		      \item \texttt{!}
		      \item \texttt{==}
		      \item \texttt{=>}
	      \end{enumerate}

	\item Which of these is valid string interpolation?
	      \begin{enumerate}[label=\Alph*)]
		      \item \texttt{"Hello \$name"}
		      \item \texttt{"Hello \{name\}"}
		      \item \texttt{"Hello + name"}
		      \item \texttt{"Hello \#name"}
	      \end{enumerate}

	\item How do you define a \textbf{getter} in Dart?
	      \begin{enumerate}[label=\Alph*)]
		      \item \texttt{int get age => \_age;}
		      \item \texttt{int age() => \_age;}
		      \item \texttt{get int age => \_age;}
		      \item \texttt{getter int age => \_age;}
	      \end{enumerate}

	\item What is the type of \texttt{var x = 3.14;}?
	      \begin{enumerate}[label=\Alph*)]
		      \item \texttt{int}
		      \item \texttt{double}
		      \item \texttt{num}
		      \item \texttt{var}
	      \end{enumerate}

	\item How do you catch exceptions in Dart?
	      \begin{enumerate}[label=\Alph*)]
		      \item \texttt{try \{ ... \} catch(e) \{ ... \}}
		      \item \texttt{try \{ ... \} except(e) \{ ... \}}
		      \item \texttt{try \{ ... \} error(e) \{ ... \}}
		      \item \texttt{catch \{ ... \}}
	      \end{enumerate}

	\item What is the output?
	      \begin{lstlisting}
void main() {
  var l = [1,2,3];
  l.add(4);
  print(l.length);
}
\end{lstlisting}
	      \begin{enumerate}[label=\Alph*)]
		      \item 3
		      \item 4
		      \item Error
		      \item 0
	      \end{enumerate}

	\item Which keyword makes a class \textbf{abstract}?
	      \begin{enumerate}[label=\Alph*)]
		      \item \texttt{abstract}
		      \item \texttt{interface}
		      \item \texttt{final}
		      \item \texttt{virtual}
	      \end{enumerate}

	\item How do you mark a parameter as \textbf{required} in a named parameter?
	      \begin{enumerate}[label=\Alph*)]
		      \item \texttt{void f(\{required int x\}) \{\}}
		      \item \texttt{void f([required int x]) \{\}}
		      \item \texttt{void f(int x!) \{\}}
		      \item \texttt{void f(required int x) \{\}}
	      \end{enumerate}

	\item How do you call a superclass constructor?
	      \begin{enumerate}[label=\Alph*)]
		      \item \texttt{super()}
		      \item \texttt{base()}
		      \item \texttt{parent()}
		      \item \texttt{this()}
	      \end{enumerate}

	\item How do you create a \textbf{constant list}?
	      \begin{enumerate}[label=\Alph*)]
		      \item \texttt{var l = [1,2,3];}
		      \item \texttt{final l = [1,2,3];}
		      \item \texttt{const l = [1,2,3];}
		      \item \texttt{List l = [1,2,3];}
	      \end{enumerate}

	\item What is the type of \texttt{List<int>}?
	      \begin{enumerate}[label=\Alph*)]
		      \item \texttt{dynamic}
		      \item \texttt{generic list of int}
		      \item \texttt{Set}
		      \item \texttt{Map}
	      \end{enumerate}

	\item Which function executes \textbf{after a future completes}?
	      \begin{enumerate}[label=\Alph*)]
		      \item \texttt{then()}
		      \item \texttt{catch()}
		      \item \texttt{async()}
		      \item \texttt{await()}
	      \end{enumerate}

	\item Which of the following \textbf{iterates a map} correctly?
	      \begin{enumerate}[label=\Alph*)]
		      \item \texttt{for(var k in map) \{\}}
		      \item \texttt{for(var e in map.entries) \{\}}
		      \item \texttt{map.foreach((k,v)\{\});}
		      \item \texttt{map.forin()}
	      \end{enumerate}

	\item What is printed?
	      \begin{lstlisting}
void main() {
  var s = "Dart";
  print(s.substring(1,3));
}
\end{lstlisting}
	      \begin{enumerate}[label=\Alph*)]
		      \item \texttt{Da}
		      \item \texttt{ar}
		      \item \texttt{rt}
		      \item \texttt{Dar}
	      \end{enumerate}

	\item How do you make a variable \textbf{late-initialized}?
	      \begin{enumerate}[label=\Alph*)]
		      \item \texttt{late int x;}
		      \item \texttt{final x;}
		      \item \texttt{var x;}
		      \item \texttt{int? x;}
	      \end{enumerate}

	\item Which statement is true about \textbf{extension methods}?
	      \begin{enumerate}[label=\Alph*)]
		      \item Can add methods to existing classes
		      \item Can override private fields
		      \item Can change the original class
		      \item Only works with List
	      \end{enumerate}

	\item Which keyword is used to \textbf{pause a function until a Future completes}?
	      \begin{enumerate}[label=\Alph*)]
		      \item \texttt{await}
		      \item \texttt{async}
		      \item \texttt{then}
		      \item \texttt{yield}
	      \end{enumerate}

	\item Which of the following \textbf{creates a set of integers}?
	      \begin{enumerate}[label=\Alph*)]
		      \item \texttt{var s = \{1,2,3\};}
		      \item \texttt{var s = [1,2,3];}
		      \item \texttt{var s = (1,2,3);}
		      \item \texttt{var s = <1,2,3>;}
	      \end{enumerate}

	\item Which of these is a valid \textbf{cascade operator usage}?
	      \begin{enumerate}[label=\Alph*)]
		      \item \texttt{myList..add(1)..add(2);}
		      \item \texttt{myList.>add(1).>add(2);}
		      \item \texttt{myList::add(1)::add(2);}
		      \item \texttt{myList**add(1)**add(2);}
	      \end{enumerate}

\end{enumerate}

\newpage

\section*{Part 2: Exercises}

\begin{enumerate}

	\item[\textbf{E1.}] Write a Dart function \texttt{factorial(int n)} that returns the factorial of \texttt{n} using recursion.

	\item[\textbf{E2.}] Create a Dart program that reads a list of integers and prints the \textbf{largest number}.

	\item[\textbf{E3.}] Write a Dart class \texttt{Person} with \texttt{name} and \texttt{age} fields, and a method \texttt{introduce()} that prints: \texttt{"Hi, my name is <name> and I am <age> years old."}.

	\item[\textbf{E4.}] Implement a Dart function \texttt{isPalindrome(String s)} that checks if a string is a palindrome (reads the same backward as forward).

	\item[\textbf{E5.}] Write a Dart function that \textbf{reverses a List} of integers.

	\item[\textbf{E6.}] Create a Dart program that counts the \textbf{number of vowels} in a string.

	\item[\textbf{E7.}] Implement a Dart class \texttt{Rectangle} with \texttt{width} and \texttt{height}, and a method \texttt{area()} that returns the area.

	\item[\textbf{E8.}] Write a Dart function \texttt{fibonacci(int n)} that returns the nth Fibonacci number \textbf{recursively}.

	\item[\textbf{E9.}] Create a Dart program that \textbf{sorts a list of strings alphabetically}.

	\item[\textbf{E10.}] Write a Dart class \texttt{BankAccount} with methods \texttt{deposit(amount)} and \texttt{withdraw(amount)} and a \texttt{balance} field.

	\item[\textbf{E11.}] Implement a Dart function that \textbf{removes duplicates from a list}.

	\item[\textbf{E12.}] Write a Dart program that reads a list of numbers and prints the \textbf{average}.

	\item[\textbf{E13.}] Create a Dart class \texttt{Car} with \texttt{brand} and \texttt{year} fields, and override the \texttt{toString()} method.

	\item[\textbf{E14.}] Write a Dart function that \textbf{checks if a number is prime}.

\end{enumerate}


\end{document}
